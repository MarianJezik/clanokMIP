\documentclass[10pt,twoside,slovak,a4paper]{article}
\usepackage[slovak]{babel}
\usepackage[T1]{fontenc}
\usepackage[IL2]{fontenc}
\usepackage[utf8]{inputenc}
\usepackage{graphicx}
\usepackage{url}
\usepackage{hyperref}

\usepackage{cite}
\usepackage{times}

\pagestyle{headings}

\title{Psychologická perspektíva na časové dimenzie e-vzdelávania\thanks{Semestrálny projekt v predmete Metódy inžinierskej práce, ak. rok 2020/21, vedenie: Michal Hatala}}

\author{Marián Ježík\\[2pt]
	{\small Slovenská technická univerzita v Bratislave}\\
	{\small Fakulta informatiky a informačných technológií}\\
	{\small \texttt{xjezikm1@stuba.sk}}
	}

\date{\small 30. október 2020} 

\begin{document}
\maketitle
\begin{abstract}
\ldots%abstrakt
\end{abstract}
\section{Úvod}

Vo vzdelávacej teórii a praxi sa už dlho reflektovalo na psychologické perspektívy času, a preto očakávame od psychológie aby prispela nášmu pochopeniu časových aspektov učenia sa v tejto digitálnej dobe. Pochopenie ako sa učíme a ako učenie môže byť uľahčené sú kľúčom k porozumeniu časovej dimenzie e-vzdelávania. Doteraz nemáme úplné porozumenie aký má čas vplyv na používanie terchnológie a obzvlášť na jej vzdelávacie účely. Pochopenie psychologického vnímania času nám môže ponúknuť náhľad do výziev e-vzdelávania. E-vzdelávanie rozšírilo časovú dimenziu učenia, preto je vyžadovaný empirický výskum, ktorý metodicky zahŕňa čas do výskumu e-vzdelávania.

\begin{enumerate}
\item Koncept času vo vzdelávaní
\item Synchrónnosť a asynchrónnosť%clanok2
\item Psychologický čas
\item Čas a pamäť
\item Výzvy a zručnosti
	\begin{enumerate}
	\item Vnímanie času
	\item Vplyv kontextu na učenie
	\item Manažment retrospektívnych a prospektívnych prvkov času
	\item Prezentácia a použitie času
	\item Distribuované poznanie
	\end{enumerate}
\item Zručnosti a psychologická infraštruktúra
\item Dôležitosť zručností
\item Záver
\end{enumerate}

\section{Koncept času vo vzdelávaní}

Aby sme mohli pochopiť e-vzdelávanie musíme najskôr pochopiť učenie a aký vplyv majú technológie na proces učenia. Existujú 3 pohľady na učenie a ich implementácie~\cite{murdoch32662}. Asociatívny pohľad na vzdelávanie predstavuje učenie ako aktivitu kde vedomosti sú získavané vytváraním spojení medzi rôznymi pojmami. Tento pohľad učenia je proces získavania vedomostí, kde si študenti postupne rozvíjajú hlbšie pochopenie a zručnosti. Preto sa pedagogika sústredí na rozvoj pomocou postupne sťažujúcich sa úloh a aktivít. Tiež zahŕňa stanovenie jasných cieľov. Kognitívny resp. konštruktívny pohľad považuje učenie za proces porozumenia, ktorým dosahujú študenti aktívne budovanie nových myšlienok rozvíjaním a testovaním hypotéz buď jednotlivo, alebo viac spoločensky prostredníctvom aktivít spolupráce a dialógu. Pedagogika preto zahŕňa zabezpečenie interaktívneho prostredia na budovanie vedomostí, aktivity, ktoré povzbudzujú k aktívnemu experimentovaniu, spolupráci a zdieľaniu nápadov, podpora reflexie a hodnotenie. Situačná perspektíva chápe učenie ako sociálnu prax, pomocou ktorej si žiaci rozvíjajú identitu prostredníctvom účasti v konkrétnych komunitách a praktikách \cite{10.2304/elea.2014.11.2.108}. Vo všetkých 3 prístupoch čas je chápaný tradičným objektívnym spôsobom fyzického lineárneho času. Poznáme 4 rôzne konceptualizácie času: čas ako trvanie, konkrétny moment v čase, plynutie času a zmena za čas \cite{BarberaE}. Technológia rozšírila časové obmedzenia na učenie, takže naša konceptualizácia sa tiež musí zmeniť aby sme pochopili ako študenti vnímajú čas keď používajú technológie na vzdelávanie.

\section{Synchrónnosť a asynchrónnosť e-vzdelávania}

Synchrónne učenie môže byť zložitné na organizáciu ak účastníci nachádzajú v rôznych častiach sveta a časových pásmach. Skupiny môžu priniesť stres ak napríklad pri projekte nechce žiak sklamať ostatných v skupine, alebo má problém s prezentovaním pred publikom. Na druhej strane byť časťou skupiny je prospešné. Skupina poskytuje určité bezpečie keď niekto nepozná správnu odpoveď a nieje nútený na ňu odpovedať keď na ňu odpovie niekto iný. Asynchrónne učenie je omnoho flexibilnejšie, pretože učiaci sa má omnoho viac slobody, pretože nieje viazaný žiadnymi konkrétnymi časmi a môže sa učiť podľa seba. Toto ale prináša výzvu, kde nie každý žiak má dosť sebakontroly a disciplíny na to, aby bol s takýmto spôsobom e-vzdelávania úspešný. \cite{10.2304/elea.2014.11.2.135}

\section{Psychologický čas}

V prvom rade musíme rozlíšiť medzi fyzickým časom a psychologickým vnímaním času \cite{zakay2012}. Fyzický čas si môžme predstaviť ako rovnú šípku z minulosti do budúcnosti. Fyzický čas sa dá rozdeliť do menších jednotiek, ktoré majú všeobecné pochopenie, ako napríklad deň alebo rok. Psychologický čas je výrazne odlišný. Je prerušovaný, nelineárny, závisí od kontextu a nemusí plynúť z minulosti do budúcnosti \cite{zakay2012}. Má skôr formu hyperpriestoru, ktorý je posobný s nespočetnou sieťou nelineárnych spojení, ktoré možno vidieť na internete. Rozhodujúce je, že tieto prepojenia v hyperpriestore psychologického času nie sú iba sledované, ale sú aktívne vytvárané. Hyperpriestor sa spolieha na sieť uzlov a vzájomne prepojené odkazy, ktoré poskytujú obrovskú slobodu voľby, pokiaľ ide o to, ktoré spojenia sa vytvoria. Podľa konekcionistických modelov má každý uzol v sieti pravdepodobnosť odoslania signálu do iného uzla v ktoromkoľvek okamihu \cite{rumelhart}. \textit{Informácie do istej miery nespočívajú v samotnom uzle, ale v sile alebo slabosti súvisiacich prepojení medzi uzlami. Pokiaľ ide o spracovanie, neexistuje sériové označenie „spätne“ alebo „dopredu“, skôr existujú rôzne úrovne aktivácie v celej sieti naraz. Ak sa domnievame, že ľudská skúsenosť a pamäť sú znázornené podobným spôsobom, potom sa časom pravdepodobne formujú pomocou sily asociovaných spojení, ktoré si každý jedinec vytvára svojím jedinečným spôsobom. Zážitok z času sa líši v závislosti od konkrétnych kognitívnych, psychosociálnych a emocionálnych faktorov. Napríklad spomienky, skúsenosti a preferencie jednej osoby budú odlišovať ich prežívanie času od inej osoby. Ľudská skúsenosť sa navyše nevyskytuje vo vákuu: skúsenosť s časom je významne ovplyvnená kontextom} \cite{10.2304/elea.2014.11.2.108}. V popredí je niekoľko teórií zaoberajúcich sa časom a jeho psychologíckého vnímania. Jedna z nich je teória sily, ktorá tvdí že sila spomienky sa postupom času zoslabuje \cite{hinrichs}, zatiaľ čo iné teórie tvrdia že kódovanie ovplyvňuje efektivitu pamäte \cite{murdock} až po modernú teóriu "attentional gate model". Podľa tejto teórie alokujeme zdroje pozornosti na budúce udalosti - keď je brána otvorená, vnímame viac časových signálov. Čím viac signálov, tým dlhšie je psychologické trvanie času \cite{zakay2000}. Preto napríklad čakanie s nejakým spôsobom rozptýlenia sa zdá kratšie ako bez neho.

\section{Vplyv kontextu na učenie}
\section{Čas a pamäť}
\section{Výzvy a zručnosti}
	\subsection{Vnímanie času}
	\subsection{Vplyv kontextu na učenie}
	\subsection{Manažment retrospektívnych a prospektívnych prvkov času}
	\subsection{Prezentácia a použitie času}
	\subsection{Distribuované poznanie}
\section{Zručnosti a psychologická infraštruktúra}
\section{Dôležitosť zručností}
\section{Záver}

\bibliography{zdroje}
\bibliographystyle{plain}

\end{document}